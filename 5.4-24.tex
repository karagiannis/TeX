\input tugboat.sty
\input macros
\pageno = 1

\noindent \bf 5.4-24\rm ~I en luftbehandlingsanl\"aggning avser man
att bereda $\dot{V}=11\ 250\rm{m}^3$/h luft med relativ
fuktighet $\varphi_{m}=0.4$ och $t_{m}=21^{\rm{o}}$C. F\"or
att uppn\aa\ detta blandas uteluft av $t_{o}=0^{\rm{o}}$C och
$\varphi_{o}=0.7$ med fr\aa nluft av $t_{2}=24^{\rm{o}}$C och
$\varphi_{2}=0.5$. Barometerst\aa ndet \"ar 1013 mbar.
Ber\"akna a) den temperatur till vilken
uteluften m\aa ste v\"armas f\"or att blandningstilst\aa ndet
skall kunna uppn\aa s; b) f\"orh\aa llandet
mellan volym-\hfill\break fl\"oden av uteluft och fr\aa nluft;
c) den f\"or uteluftens uppv\"armning erforderliga
v\"armeeffekten

\medskip
%\halign{ 	   #&#&#  			 			& \quad\hfil	#&# 							\cr
%\it{Givet}\rm 	&{}&{} 						& 			    {}&\it{S\"okt}\rm	\cr
%%$\dot{V}$		&=&$11\ 250{\rm{m}}^3$/h    & 		{}&{}%$\t_{o}$ 					\cr
%$\varphi_{m}$	&=&0.4	     				& 				{}&$\dot{V_{o}}$ 	\cr
%$t_{m}$			&=&$21^{\rm{o}}$C   	& 				{}&$\Delta i_{o}$	\cr
%$t_{o}$			&=&$0^{\rm{o}}$C    	& 				{}&{} 				\cr
%$\varphi_{o}$	&=&0.7						& 				{}&{} 				\cr
%$t_{2}$			&=&$24^{\rm{o}}$C   	& 				{}&{} 				\cr
%$\varphi_{2}$	&=&0.5						& 				{}&{} 				\cr
%p				&=&1.013 bar				& 				{}&{} 				\cr

%}

%\halign{\hfil#\unskip&{}#{}&#\hfil & \quad\hfil     #&#\hfil                 \cr
%\it Givet   	&{}&{}              &               {}&\it S\"okt   \cr
%$\varphi_{m}$   &=&0.4             &               {}&$\dot{V_{o}}$    \cr
%$t_{m}$         &=&$21^{\rm o}$C   &               {}&$\Delta i_{o}$   \cr
%$t_{o}$         &=&$0^{\rm o}$C    &               {}&$t_{o}'$        \cr
%$\varphi_{o}$   &=&0.7             &               {}&{}               \cr
%$t_{2}$         &=&$24^{\rm o}$C   &               {}&{}               \cr
%$\varphi_{2}$   &=&0.5             &               {}&{}               \cr
%p               &=&1.013 bar       &               {}&{}               \cr
%}


%\halign{	   #&#&#               	& \quad\hfil     #&#\hfil           \cr
%\it Givet  		&{}&            &               {}&\it S\"okt   	\cr
%$\varphi_{m}$ 	&=&0.4              &               {}&$\dot{V_{o}}$    \cr
%$t_{m}$      	&=&$21^{\rm o}$C   	&               {}&$\Delta i_{o}$   \cr
%$t_{o}$         &=&$0^{\rm o}$C    &               {}&$t_{o}'$        	\cr
%$\varphi_{o}$   &=&0.7             &               {}&{}               \cr
%$t_{2}$         &=&$24^{\rm o}$C   &               {}&{}               \cr
%$\varphi_{2}$   &=&0.5             &               {}&{}               \cr
%p               &=&1.013 bar       &               {}&{}               \cr
%}

%\halign{\hfil#\unskip&${}#{}$&#\unskip\hfil   & \quad\hfil    #&#\unskip\hfil   \cr
%\multispan{2}{\it Givet}&{}              &               {}&\it S\"okt   \cr
%$\varphi_{m}$   &=&0.4              &               {}&$\dot{V_{o}}$    \cr
%$t_{m}$         &=&$21^{\rm o}$C   &               {}&$\Delta i_{o}$   \cr
%$t_{o}$         &=&$0^{\rm o}$C    &               {}&$t_{o}'$        \cr
%$\varphi_{o}$   &=&0.7              &               {}&{}               \cr
%$t_{2}$         &=&$24^{\rm o}$C   &               {}&{}               \cr
%$\varphi_{2}$   &=&0.5              &               {}&{}               \cr
%p               &=&1.013 bar        &               {}&{}               \cr
%}


%\bigskip

%\halign{#\unskip\hfil&${}#{}$&#\hfil      	& \quad\hfil     #&#		\hfil    	\cr
%\multispan{2}{\it Givet}&{}              	&               {}&\it S\"okt   		\cr
%$\varphi_{m}$   &=&0.4              		&               {}&$\dot{V_{o}}$    	\cr
%$t_{m}$         &=&$21^{\rm o}$C   		&               {}&$\Delta i_{o}$   	\cr
%$t_{o}$         &=&$0^{\rm o}$C    		&               {}&$t_{o}'$        		\cr
%$\varphi_{o}$   &=&0.7              		&               {}&{}               	\cr
%$t_{2}$         &=&$24^{\rm o}$C   		&               {}&{}               	\cr
%$\varphi_{2}$   &=&0.5              		&               {}&{}               	\cr
%p               &=&1.013 bar        		&               {}&{}               	\cr
%}
\bigskip %Denna är samma som ovan men utan {}
\halign{#\unskip\hfil &#&#	\hfil      		& \quad\hfil     #&#		\hfil    				\cr
\multispan{2}{\it Givet}&	              	&               {}&\it S\"okt   					\cr
$\dot{V_{m}}$		&=&$11\ 250 \rm{m}^3$/h &               {}&$t_{o}'$    						\cr
$\varphi_{m}$   &=&0.4              		&               {}&$\Delta i_{o}$   				\cr
$t_{m}$         &=&$21^{\rm o}$C   			&               {}&$\dot{V_{o}}/\dot{V_{2}}$  		\cr
$t_{o}$         &=&$0^{\rm o}$C    			&               {}&{}        						\cr
$\varphi_{o}$   &=&0.7              		&               {}&{}               				\cr
$t_{2}$         &=&$24^{\rm o}$C   			&               {}&{}               				\cr
$\varphi_{2}$   &=&0.5              		&               {}&{}               				\cr
p               &=&1.013 bar        		&               {}&{}               				\cr
}

%\bigskip

%\halign{#\unskip\hfil                	& \quad\hfil    #&#\unskip\hfil      \cr
%\it Givet               				&               {}&\it S\"okt   	\cr
%$\varphi_{m} =0.4 $             		&               {}&$\dot{V_{o}}$    \cr
%$t_{m}       =21^{\rm o}{\rm C}$   	&               {}&$\Delta i_{o}$   \cr
%$t_{o}       =0^{\rm o}{\rm C}$    	&               {}&$t_{o}'$        	\cr
%$\varphi_{o} =0.7    $          		&               {}&{}               \cr
%$t_{2}       =24^{\rm o}{\rm C}$   	&               {}&{}               \cr
%$\varphi_{2} =0.5$              		&               {}&{}               \cr
%${\rm p}     =1.013 \,{\rm bar}$       &               {}&{}               \cr
%}

\medskip
\noindent Blandningsfl\"odet $\dot{V_{m}}$ \"ar givet, vilket enligt allm\"anna gaslagen \"ar
$$\dot{V}_{m} = {{\dot{m}_{m}\cdot R_{m}\cdot T_{m}}\over p } \eqno (1)$$

\medskip
\noindent Blandningsluften \"ar fuktig s\aa\ d\"arf\"or best\aa massfl\"odet $\dot{m}_{m}$ av massfl\"odet torrluft och vatten.
$$\eqalignno{\dot{m}_{m}&=\dot{m}_{mL}+\dot{m}_{m\HTWO} \cr
					 {}&=\dot{m}_{mL}+x_{m}\cdot\dot{m}_{mL} & (2) \cr
}
$$

\medskip
\noindent Blandningsluften torra luftf\"ode respektive dess vatteninneh\aa ll kommer fr\aa n fr\aa nluften och
tillluften
$$\eqalignno{m_{mL}&=m_{oL}+m_{2L}  &(3)\cr
	   m_{m\HTWO}&=m_{o\HTWO}+m_{2\HTWO} &(4)\cr
}$$

\medskip
\noindent Enligt teorin f\"or blandning s\aa\ m\aa ste punkterna f\"or blandningsluften, till-luften och fr\aa nluften
ligga p\aa \hfill\break en r\"at linje i Mollier-diagrammet. Uteluften m\aa ste ges energitillskottet $\Delta i_{o}$
s\aa dant att temperaturen $t_{o}'$ uppn\aa s. Proportionerna fr\aa nluft och tilluft best\"a-mmer \"aven vattenm\"angden
$x_m$ kg \HTWOO per kg torrluft.
$$\eqalignno{
i_{m} &= {{m_{oL}\cdot i'_{o} + m_{2L}\cdot i_{2}}\over{m_{oL}+m_{2L}}}  &(5)\cr
x_{m} &= {{m_{oL}\cdot x_{o} + m_{2L}\cdot x_{2}}\over{m_{oL}+m_{2L}}} 	 &(6)\cr
}
$$

\medskip 
\noindent F\"or att kunna ber\"akna $t_{o}'$ och $i'_{o}$ numeriskt s\aa\ m\aa ste vi l\"osa ut $i'_{o}$ ur (5) men 
med $m_{oL}$ och $m_{2L}$ i uttryckt n\aa got som givits ur problemformuleringen.
Vi kan dock inte
l\"osa ut $\dot{m}_{m}$ ur (1) d\"arför att vi inte k\"anner $R_{m}$ som beror blir olika beroende p\aa\ 
proportionerna torrluft och vatten i blandningen. Vi har dock en graf \"over densiteten p\aa\ sid. 455
och det m\aa ste g\"alla att 
$$\rho = {m\over V} = {\dot{m}\over\dot{V}}={\dot{m}_m\over\dot{V}_m} = {{\dot{m}_{m\HTWO}+\dot{m}_{mL}}\over\dot{V}_m} \eqno (7)$$
s\aa\ ett delm\aa l m\aa ste vara att uttrycka $m_{oL}$ och $m_{2L}$ som funktion av $m_{m}$

Vi b\"or kunna skriva om uttrycket f\"or $x_m$ utan att beh\"ova skriva $\dot{x}_m$ d\"arf\"or att dess proportioner
m\aa ste vara en konstant.
$$\eqalignno{
x_{m}&= {{\dot{m}_{oL}\cdot x_{o} + \dot{m}_{2L}\cdot x_{2}}\over{\dot{m}_{oL}+\dot{m}_{2L}}} & (8)\cr
(\dot{m}_{oL}+\dot{m}_{2L})\cdot x_{m}&= \dot{m}_{oL}\cdot x_{o} + \dot{m}_{2L}\cdot x_{2}\cr
}$$

\medskip
\noindent Samlar ihop $\dot{m}_{oL}$ p\aa\ v\"anster sida och $\dot{m}_{2L}$ p\aa\ h\"oger
$$\eqalignno{
\dot{m}_{oL}\cdot x_{m}-\dot{m}_{oL}\cdot x_{o} &=\dot{m}_{2L}\cdot x_{2}-\dot{m}_{2L})\cdot x_{m}\cr
\dot{m}_{oL}\cdot ( x_{m}-x_{o} )&=\dot{m}_{2L}\cdot(x_{2}-x_{m})\cr
\dot{m}_{oL}&=\dot{m}_{2L}\cdot{{x_{2}-x_{m}}\over {x_{m}-x_{o}}}   &(9)\cr}$$

\medskip
\noindent Vi har en relation mellan $\dot{m}_{oL}$ och $\dot{m}_{2L}$ i (8) men vi beh\"over uttrycka b\aa da dessa
i n\aa got som givits ur problemst\"allningen. Volymfl\"odet per tidsenhet $\dot{V}_m$ har givits som vi kan relatera till
massfl\"odet $\dot{m}_m$.
%
$\dot{m}_{m}$ har givits indirekt genom (1).

\medskip
\noindent Vi l\"oser ut $m_{mL}$ ur (2) och substituerar i (3)
$$\eqalignno{{\dot{m}_m\over(1+x_m)}&= \dot{m}_{oL} + \dot{m}_{2L}\cr
\dot{m}_m&=(\dot{m}_{oL} + \dot{m}_{2L})\cdot (1+x_m) \cr
}$$

\medskip
\noindent Vi anv\"ander (8) f\"or att f\aa\ bort $\dot{m}_{oL}$ och kommer d\aa\ f\aa\ 
en relation mellan $\dot{m}_m$ och $\dot{m}_{2L}$. Anv\"ander vi sedan (8) s\aa\ f\aa r vi också
en relation mellan $\dot{m}_{oL} $ och $\dot{m}_m$ och har d\aa\ f\aa t vad vi var ute efter, n\"amligen
att uttrycka  $\dot{m}_{oL} $ och $\dot{m}_{2L}$ i n\aa got bekant.
$$\eqalignno{
\dot{m}_m&=\Bigl(\dot{m}_{2L}\cdot{{x_{2}-x_{m}}\over {x_{m}-x_{o}}} + \dot{m}_{2L}\Bigr)\cdot (1+x_m)\cr
\dot{m}_m&=\dot{m}_{2L}\cdot\Bigl({{x_{2}-x_{m}}\over {x_{m}-x_{o}}}+1 \Bigr)\cdot (1+x_m) & (10) \cr
}$$ 

\medskip
\noindent Nu kan vi v\"anda p\aa\  (10) l\"osa ut  $\dot{m}_{2L}$ 
$$\eqalignno{
\dot{m}_{2L}&={\dot{m}_m \over {\Bigl({{x_{2}-x_{m}}\over {x_{m}-x_{o}}}+1 \Bigr)\cdot (1+x_m)} } & (11) \cr
}$$ 

\medskip
\noindent Vi anv\"ander nu (8) p\aa\  (11)
$$\eqalignno{
\dot{m}_{oL}&={{{{x_{2}-x_{m}}\over {x_{m}-x_{o}}}}\cdot {\dot{m}_m \over {\Bigl({{x_{2}-x_{m}}\over {x_{m}-x_{o}}}+1 \Bigr)\cdot (1+x_m)} } }& (12)\cr
}$$ 

\medskip
\noindent Grafen på sid. 455 ger att densiteten $\rho$ f\"or den fuktiga luften \"ar ca 1.2kg/$\rm{m}^3$

$$\dot{m}_m = \rho \cdot \dot{V}_{m} = 1.2\cdot{11250\over 3600}= 3.75 \rm{kg/m}^3 \eqno(13)$$

\medskip
\noindent H\"arifr\aa n \"ar det en enkel sak att f\aa\  numeriska v\"arden p\aa\ $x_{o},x_{2}$ och $x_m$
som beh\"ovs f\"or att f\aa\ siffror p\aa\ ${m}_{oL}$ och ${m}_{2L}$ som beh\"ovs f\"or att f\aa\ en siffra
p\aa\ $i'_{o}$ och slutligen temperaturen $t_{o}'$. Detta g\"ors med ekv. $(5.4.4-6\rm{a})$ p\aa\ sid. 455.
$$\eqalignno{
x_{o}&=0.621\cdot{{\varphi_{o}\cdot p_{\HTWO}"(0^{\rm{o}}\rm{C})}\over {p-{\varphi_{o}\cdot p_{\HTWO}"(0^{\rm{o}}\rm{C})}}}\cr
x_{o}&=0.621\cdot{{0.7\cdot 0.006107}\over {1.013 -0.7\cdot  0.006107 }} \cr
x_{o}&=0.0026318 &(13) \cr
}$$

$$\eqalignno{
x_{2}&=0.621\cdot{{\varphi_{2}\cdot p_{\HTWO}"(24^{\rm{o}}\rm{C})}\over {p-{\varphi_{2}\cdot p_{\HTWO}"(24^{\rm{o}}\rm{C})}}}\cr
x_{2}&=0.621\cdot{{0.5\cdot 0.029824}\over {1.013 -0.5\cdot  0.02984 }} \cr
x_{2}&=0.0092781 &(14) \cr
}$$

$$\eqalignno{
x_{m}&=0.621\cdot{{\varphi_{m}\cdot p_{\HTWO}"(21^{\rm{o}}\rm{C})}\over {p-{\varphi_{m}\cdot p_{\HTWO}"(21^{\rm{o}}\rm{C})}}}\cr
x_{m}&=0.621\cdot{{0.4\cdot 0.024855}\over {1.013 -0.4\cdot  0.024855 }} \cr
x_{m}&=0.0061552 &(15) \cr
}$$

\medskip
\noindent Nu kan $\dot{m}_{oL}$ och $\dot{m}_{oL}$ ber\"aknas
$$\eqalignno{
\dot{m}_{2L}&={\dot{m}_m \over {\Bigl({{x_{2}-x_{m}}\over {x_{m}-x_{o}}}+1 \Bigr)\cdot (1+x_m)} } & (11) \cr
          {}&={3.75 \over{\Bigl({{0.0092781-0.0061552}\over {0.0061552-0.0026318}}+1 \Bigr)\cdot (1+0.0061552)} } & \cr
		  {}&=1.9758 \cr
}$$ 

$$\eqalignno{
\dot{m}_{oL}&=\dot{m}_{2L}\cdot{{x_{2}-x_{m}}\over {x_{m}-x_{o}}}   & (8)\cr
			&=1.9758\cdot{{0.0092781-0.0061552}\over {0.0061552-0.0026318}}\cr
			&=1.7512\cr			
}$$

\medskip
\noindent Eftersom (5) \"ar en relatio innehållandes kvoter av massor s\aa\ m\aa ste samma
likhet g\"alla om kvoterna \"ar uttrycks som kvoter mellan massfl\"oden om dessa inte varierar med tiden. 
Masskvoterna i h\"oger-ledet av (5) m\aa ste ju g\"alla
f\"or alla tidpunkter. F\"olj-ande m\aa ste d\"arf\"or ocks\aa\  vara sant 
$$\eqalignno{
i_{m} &= {{\dot{m}_{oL}\cdot i'_{o} + \dot{m}_{2L}\cdot i_{2}}\over{\dot{m}_{oL}+\dot{m}_{2L}}}  &(14)\cr
}
$$

\medskip
\noindent Vi ber\"aknar f\"orst $i_{m}$ och $i_{2}$ enligt ekv. $5.4.4-9$ i boken f\"or att sedan l\"osa ut
$i'_{o}$ som d\"arefter kommer ge oss $t_{o}'$ med hj\"alp av samma formel
$$\eqalignno{
i_{m} &= t + x_{m}\cdot(2500 + 1.86\cdot t_{m})\cr
      &= 21 + 0.0061552\cdot (2500+1.86\cdot 21) \cr
      &= 36.628 \rm{\ kJ\ per\ kg\ torrluft}
}
$$

$$\eqalignno{
i_{2} &= t + x_{2}\cdot(2500 + 1.86\cdot t_{2})\cr
      &= 24 + 0.0092781\cdot (2500+1.86\cdot 24) \cr
      &= 47.609 \rm{\ kJ\ per\ kg\ torrluft}
}
$$

\medskip
\noindent Nu kan $i'_{o}$ l\"osas ut. Multiplcera (14) med n\"amnaren och subtrahera $\dot{m}_{2L}\cdot i_{2}$
%$$\eqalignno{
%i_{m} &= {{\dot{m}_{oL}\cdot i'_{o} + \dot{m}_{2L}\cdot i_{2}}\over{\dot{m}_{oL}+\dot{m}_{2L}}}  &(14)\cr
%}
%$$
$$\eqalignno{
i_{m}\cdot({\dot{m}_{oL}+\dot{m}_{2L}})- \dot{m}_{2L}\cdot i_{2} &= {\dot{m}_{oL}\cdot i'_{o} }\cr
}
$$
$$\eqalignno{
 i'_{o} &= {{i_{m}\cdot({\dot{m}_{oL}+\dot{m}_{2L}})- \dot{m}_{2L}\cdot i_{2}}\over  \dot{m}_{oL}}\cr
        &= {{36.628\cdot({1.7512+1.9758})- 1.9758\cdot 47.609}\over  1.7512}\cr
		&=24.239 \rm{\ kJ\ per\ kg\ torrluft} \cr
}
$$
Vi v\"ander p\aa\ ekvation ekv. $5.4.4-9$ i boken f\"or att sedan l\"osa ut $t_{o}'$.
$$\eqalignno{
i'_{o} &= t_{o}' + x_{o}\cdot(2500 + 1.86\cdot t_{o}')\cr
 i'_{o}- x_{o}\cdot 2500  &=  t_{o}' + x_{o}\cdot 1.86\cdot t_{o}'\cr
 t_{o}' &= {{i'_{o}- x_{o}\cdot 2500}\over {1+x_{o}\cdot 1.86}}\cr
 		&= {{24.239 - 0.0026318\cdot 2500}\over {1+0.0026318\cdot 1.86}}\cr
		&=17.573^{ \rm{o}}\rm{C}
}
$$

\medskip
\noindent Svaret p\aa\ fr\aa ga a) \"ar $17.573^{ \rm{o}}\rm{C}$. Fr\aa ga c) avseende
den erfoderliga v\"armeeffekten f\aa s som skillnaden mellan $i'_{o}$ och $i_{o}$ multiplicerat
med torrluftsfl\"odet $\dot{m}_{oL}$. Vi beh\"over r\"akna ut $i_o$

$$\eqalignno{
i_{o} &= t_{o} + x_{o}\cdot(2500 + 1.86\cdot t_{o})\cr
      &= 0 +0.0026318\cdot(2500 +1.86\cdot 0)\cr
	  &=6.5795 \rm{\ kJ\ per\ kg\ torrluft} \cr
}
$$
$$\eqalignno{
\dot{Q} &= \dot{m}_{oL}\cdot(i'_{o}-i_{o})\cr
   		&=1.7512\cdot(24.239-6.5795)\cr
		&=30.925 \rm{\ kW} \cr
}
$$

 
 




\bye
